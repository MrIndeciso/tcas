\begin{figure}
	\centering
	\begin{lstlisting}[style=CStyle]
		#ifndef EXPR_STRUCTS_H
		#define EXPR_STRUCTS_H
		
		#include <stdio.h>
		#include <stddef.h>
		#include <gmp.h>
		#include <mpfr.h>
		
		enum VAL_TYPE {
			INT = 0b001,
			RATIONAL = 0b010,
			FLOAT = 0b100,
		};
		
		struct expr_tree_head {
			struct expr_tree_link *head;
		};
		
		struct expr_tree_op {
			enum OPERATOR_TYPE type;
			size_t arg_count;
			struct expr_tree_link **args;
		};
		
		struct expr_tree_val {
			enum VAL_TYPE type;
			union expr_tree_val_ref *val;
		};
		
		struct expr_tree_sym {
			char sign;
			char representation;
		};
		
		struct expr_tree_link {
			enum LINKED_TYPE type;
			union expr_tree_ptr *ptr;
		};
		
		union expr_tree_ptr {
			struct expr_tree_val *val;
			struct expr_tree_op *op;
			struct expr_tree_sym *sym;
		};
		
		union expr_tree_val_ref {
			mpz_t int_val;
			mpq_t rational_val;
			mpfr_t fp_val;
		};
		
		
		#endif
		
	\end{lstlisting}
	\caption{The code that defines the concrete syntax tree} \label{code:expr_structs}
\end{figure}

\begin{figure}
	\centering
	\begin{lstlisting}[style=CStyle]
		struct ps_leadterm {
			int term_num;
			struct expr_tree_link *coeff;
			struct expr_tree_val *exp;
		};
	\end{lstlisting}
	\caption{How the first term of the power series is represented} \label{code:ps_structs}
\end{figure}

\begin{figure}
	\centering
	\begin{lstlisting}[style=CStyle]
		struct expr_tree_link *compute_gruntz_result(struct expr_tree_link *link) {
			struct ps_leadterm *term = _recursive_compute_lt(link);
			
			while (term->exp->type != INT) { //Something is wrong here, next term
				term = _recursive_lt_next(term, link);
			}
		
			if (mpz_cmp_si(term->exp->val->int_val, 0) > 0) { //Exp > 0
				return parse_expr("+infinity", NULL);
			} else if (mpz_cmp_si(term->exp->val->int_val, 0) == 0) { //Exp == 0
				return gruntz_eval(term->coeff);
			} else { //Exp < 0
				return parse_expr("0", NULL);
			}
		}
	\end{lstlisting}
	\caption{How the result is evaluated} \label{code:ps_result}
\end{figure}

\begin{figure}
	\centering
	\begin{lstlisting}[style=CStyle]
		struct gruntz_mrv {
			size_t count;
			struct gruntz_expr **expr;
		};
		
		struct gruntz_expr {
			struct expr_tree_link *expr;
		};
	\end{lstlisting}
	\caption{How the MRV structs are defined} \label{code:mrv_structs}
\end{figure}

\begin{figure}
	\centering
	\begin{lstlisting}[style=CStyle]
		struct gruntz_mrv *_mrv_op(struct expr_tree_link *link) {
			switch (link->ptr->op->type) {
				case PLUS:
				case MINUS:
				case TIMES:
				case DIVIDE:
					return _mrv_max(_mrv_generic(link->ptr->op->args[0]), _mrv_generic(link->ptr->op->args[1]));
					break;
				case SQRT:
				case SIN:
				case COS:
				case TAN:
				case LN:
				case LOG10:
				case LOG2:
					return _mrv_generic(link->ptr->op->args[0]);
					break;
				case ROOT:
				case POWER:
					assert(link->ptr->op->args[1]->type == VALUE);
					return _mrv_generic(link->ptr->op->args[0]);
					break;
				case EXP:
					return _mrv_exp(link);
					break;
				default:
					assert(0);
			}
		}
	\end{lstlisting}
	\caption{\textit{\_mrv\_op}, refer to \cref{code:expr_structs} and \cref{code:mrv_structs} for how the data types are implemented} \label{code:mrv_op}
\end{figure}