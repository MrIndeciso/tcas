% !TeX encoding = UTF-8
% !TeX spellcheck = en_GB
% !TeX root = tcas.tex
\documentclass{article}
\usepackage[utf8]{inputenc}
\usepackage{authblk}
\usepackage{setspace}
\usepackage[margin=3.5cm]{geometry}
\usepackage{graphicx}
\usepackage{subcaption}
\usepackage{amsmath}

\usepackage{biblatex}
\addbibresource{tcas.bib}

\title{T.C.A.S.: TCAS Can Always Solve \\
		\large A Numerical Analysis Method for Limit Evaluation}
\author{Roberto Alessandro Bertolini}

\date{}

\affil{Liceo Nervi Ferrari - Morbegno}
\onehalfspacing

\begin{document}
	\maketitle
	
	\begin{abstract}
		
	\end{abstract}

	\tableofcontents
	
	\newpage	
	
	\section{Introduction}
	
	\subsection{The Limit of a Function}
	
	\subsection{Polish Notation}
	
	\subsection{The Shortcomings of the Naive Approach}
	
	\section{The Gruntz Algorithm}
	
	\subsection{The Most Rapidly Varying Subexpression}
	
	\subsection{Power Series Representation}
	
	\subsection{Caveats and Limitations}
	
	\section{TCAS}
	
	\subsection{Expression Parsing}
	
	\subsection{Finding the MRV}
	
	\subsection{Evaluating the Result}
	
	\subsection{Performance Considerations}
	
	\printbibliography
	
\end{document}