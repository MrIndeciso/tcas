% !TeX encoding = UTF-8
% !TeX spellcheck = en_GB
% !TeX root = tcas.tex
\documentclass{article}
\usepackage[utf8]{inputenc}
\usepackage{authblk}
\usepackage{setspace}
\usepackage[margin=3.5cm]{geometry}
\usepackage{graphicx}
\usepackage{subcaption}
\usepackage{amsmath}
\usepackage{amsthm}
\usepackage{biblatex}
\addbibresource{tcas.bib}

\title{\textbf{T.C.A.S.}: \textbf{T}CAS \textbf{C}an \textbf{A}lways \textbf{S}olve \\
		\large Numerical Analysis: a Concrete Computational Approach to Limit Evaluation}
\author{Roberto Alessandro Bertolini}

\date{}

\affil{Liceo Nervi Ferrari - Morbegno}
\onehalfspacing

\theoremstyle{plain}
\newtheorem{thm}{Theorem}

\theoremstyle{definition}
\newtheorem{defn}[thm]{Definition}

\begin{document}
	\maketitle
	
	\begin{abstract}
		Evaluating limits by hand can become a trivial task with a bit of exercise, but a regular computer is generally incapable of proceeding intuitively and needs a reliable algorithm in order to be able to reach consistently the same result. 
		While some purely heuristic or naive approaches might, at first glance, seem good enough, they tend to quickly fall apart in real conditions. TCAS is a portable universal C implementation of the Gruntz Algorithm \cite{gruntz}, which at the present day is the most efficient and reliable way of evaluating limits in the exp-log field of operations.
	\end{abstract}

	\tableofcontents
	
	\newpage	
	
	\section{Introduction}
	
	\subsection{The Limit of a Function}
	
	The limit of the function $ f: \mathbf{R} \rightarrow \mathbf{R} $ is defined as the following:
	
	\begin{defn}
		\[ 
		\lim_{x \rightarrow x_{0}}{f(x) = l} 
		 \]
		 
		 If and only if \( 
		 \forall \epsilon > 0 \enspace \exists \enspace \delta(\epsilon) \mid \forall x \in D_{f}, 0 < \mid x - x_{0} \mid < \delta \implies \mid f(x) - l \mid < \epsilon
		  \)
	\end{defn}
	
	\subsection{Polish Notation}
	
	
	
	\subsection{The Shortcomings of the Naive Approach}
	
	\section{The Gruntz Algorithm}
	
	\subsection{The Most Rapidly Varying Subexpression}
	
	\subsection{Power Series Representation}
	
	\subsection{Caveats and Limitations}
	
	\section{T.C.A.S.}
	
	\subsection{Expression Parsing}
	
	\subsection{Finding the MRV}
	
	\subsection{Evaluating the Result}
	
	\subsection{Performance Considerations}
	
	\printbibliography
	
\end{document}